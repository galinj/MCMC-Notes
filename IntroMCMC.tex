\documentclass[12pt]{article}

\usepackage{geometry}
\geometry{hmargin=.75in,vmargin={1.25in,1.25in},nohead,footskip=0.5in}
\renewcommand{\baselinestretch}{1.5}
\setlength{\baselineskip}{0.4in} \setlength{\parskip}{.05in}

%% FIGURES AND TABLES
\usepackage{graphicx}
\usepackage{floatrow}

\usepackage[numbers]{natbib}
\usepackage{algorithm}
\usepackage{algpseudocode}
\usepackage{url}
\usepackage{hyperref}
\usepackage[dvipsnames]{xcolor}
\usepackage{amsmath, amssymb, amsthm}
\usepackage{verbatim}
\usepackage{subfig}

\theoremstyle{plain}
\newtheorem{thm}{Theorem}[section]
\newtheorem{conj}{Conjecture}[section]
\newtheorem{lem}[thm]{Lemma}
\newtheorem{prop}[thm]{Proposition}
\newtheorem{cor}{Corollary}[section]

\theoremstyle{definition}
\newtheorem{defn}{Definition}[section]
\newtheorem{assmp}{Assumption}

\theoremstyle{remark}
\newtheorem{remark}{Remark}[section]
\newtheorem{example}{Example}[section]
\newtheorem{hw}{Exercise}[section]

\newcommand{\argmin}{\operatorname{argmin}}

\newcommand{\df}{\mathrm{d}}
\newcommand{\ds}{\displaystyle}

\newcommand{\mF}{\mathcal{F}}

\newcommand{\X}{\mathsf{X}}
\newcommand{\Y}{\mathsf{Y}}
\newcommand{\B}{\mathcal{B}}
\newcommand{\BX}{{\cal B}(\mathsf{X})}

\newcommand{\real}{\mathbb{R}}
\newcommand{\exreal}{\overline{\real}}
\newcommand{\tr}{\mbox{tr}}



\title{Markov Chain Monte Carlo}
\author{Galin L. Jones\\
%{}\\
{\small School of Statistics}\\
{\small University of Minnesota}}
\date{Draft: \today}
\begin{document}

\maketitle

\tableofcontents

\section{Introduction}
\label{mcmc:sec:intro}

Later.

\section{Markov Chains}
\label{mcmc:sec:markov}

Let $(\X, \B)$ be a measurable space.  A sequence of
$\X$-valued random variables $\{X_1, X_2, X_3, \ldots \}$ is a Markov
chain if for all $g$
\[
E\left[ g(X_{n+1}, X_{n+2}, \ldots) \mid X_n, X_{n-1}, \ldots, X_1
\right] = E\left[ g(X_{n+1}, X_{n+2}, \ldots) \mid X_n\right].
\]

Then $P$ is a {\em Markov kernel}
if $P : \X \times \mathcal{B} \to \real$ satisfying (i) for each fixed
$x \in \X$, $P(x, \cdot)$ is a probability measure and (ii) for each fixed
$B \in \mathcal{B}$, $P(\cdot, B)$ is a measurable function.  

When $\X$ is a discrete set a Markov kernel can be represented as a
square matrix whose entries are nonnegative and whose rows sum to 1.

\begin{example}
  Suppose
  \[
    P= \begin{pmatrix}
      1/2 & 1/2 \\
      1/3 & 2/3\\
      \end{pmatrix}
  \]
  Then $P$ is  a Markov matrix on two states $\{0, 1\}$, say.  The
  first row for example, is interpreted as
  the probability of moving in one step from state 0 to state 0 is 1/2
  which is the same as the probability of moving in one step from state 0 to state 1.
 \end{example}

\begin{example}
  Let $\X=\mathbb{Z}$ and let $0 < \theta < 1$.  If $x \ge 1$, then a
  Markov kernel is defined by the matrix $P$ with elements
  \[
    P(x, x+1) = P(-x, -x-1) = \theta, \quad \quad P(x,0)=P(-x,0) = 1- \theta,
  \]
  and $P(0,1)=P(1,0) =1/2$.
\end{example}

Often $\X$ will be uncountable and $\B$ will be countably
generated.  If $\X$ is topological, then $\B$ will be the Borel
$\sigma$-algebra generated by $\X$.

\begin{example}
  \label{mcmc:ex:betawalk}
 Let $\X  = (0,1)$ and consider the Markov chain that evolves as
 follows. Draw $U \sim \text{Uniform}(0,1)$.  If $u \le 0.$5, $X_{n+1}
 \sim \text{Uniform}(0, X_n)$, but if $u > 0.5$, $X_{n+1}
 \sim \text{Uniform}(X_n, 1)$.  Then if $X_n = x$ and $B \in \B$
 \[
P(x, B) = \int_{B} \left[ \frac{1}{2} \frac{1}{x} I_{y}((0,x)) +
  \frac{1}{2} \frac{1}{1-x} I_{y}(x,1) \right] \df y .
 \]
\end{example}

In Example~\ref{mcmc:ex:betawalk}, the integrand in the Markov kernel
is a conditional density on $\X$.  This is a setting that will be
encountered repeatedly throughout.  If there is a conditional density
$k (y \mid x)$, with respect to a measure $\lambda$, such that the Markov kernel satisfies for $B \in \B$
\[
P(x, B) = \int_{B} k(y \mid x) \lambda(\df y),
\]
then say $k$ is a {\em Markov transition density}.

\begin{example}
  \label{mcmc:ex:2varGS}
Suppose $f(x,y)$ is a joint density with support $\mathbb{R}^2$ and
conditional densities $f_{X \mid Y}(x \mid y)$ and $f_{Y \mid X}(y
\mid x)$.  Then
\[
  k(x', y' \mid x, y) = f_{X \mid Y}(x' \mid y) f_{Y \mid X}(y' \mid x')
\]
is a Markov transition density.  The Markov chain evolves from
$(X_k = x, Y_k =y)$ to $(X_{k+1}, Y_{k+1})$ by drawing
$X_{k+1} \sim F_{X \mid Y} (\cdot \mid y)$ followed by
$Y_{k+1} \sim F_{Y \mid X}(\cdot \mid X_{k+1})$.  This is a special
case of the so-called two-variable Gibbs sampler.
\end{example}


Suppose $\lambda$ is a positive measure on $(\X, \B)$,  define
\begin{equation}
  \label{mcmc:eq:left}
\lambda P(B) = \int_{\X} \lambda(\df x) P(x, B) .
\end{equation}
When $\lambda$ is a probability measure, the encouraged interpretation
is that $X_{n+1} \mid X_{n} \sim P(X_{n}, \cdot)$ and
$X_{n} \sim \lambda$, the product $\lambda(\df x) P(x, \cdot)$ is the
joint distribution of $(X_n, X_{n+1})$ and $\lambda P$ is the marginal
distribution of $X_{n+1}$.

Since Markov kernels act to the left on measures~\eqref{mcmc:eq:left},
\[
P^2(x, B) = \int_{\X} P(x, \df x_k) P(x_k, B) .
\]
Continuing in this fashion obtain for every $n \ge 2$
\[
P^n(x, B) \int_{\X} P(x, \df x_k) P(x_k, \df x_{k+1}) \cdots P(x_{k +
  n -2}, B) .
\]
More generally, the so-called Chapman-Kolmogorov equations hold for $n
\ge m \ge 0$
\[
P^n(x, B) \int_{\X} P^m(x, \df y) P^{n-m} (y, B) .
\]

If $\lambda = \lambda P$, then $\lambda$ is {\em invariant} for $P$.
Notice that if $\lambda$ is invariant for $P$ and $X_n \sim \lambda$,
then $X_{n+1} \sim \lambda$.  That is, the marginal distribution does
not depend upon $n$ in which case the Markov chain is {\em
  stationary}.

{\color{blue} Come back to these examples.}

\begin{example}
\end{example}

\begin{example}
  Recall the Markov chain defined in Example~\eqref{mcmc:ex:2varGS}
\end{example}


One common way of establishing invariance of MCMC Markov chains is to
verify a {\em detailed balance condition}; see
Exercise~\ref{mcmc:hw:dbc}.  Detailed balance holds if
\begin{equation}
   \label{mcmc:eq:dbc}
   \lambda(\df x) P(x, \df y) = \lambda(\df y) P(y, \df x). 
 \end{equation}
 When $\lambda$ is a probability measure, one interpretation is that
 the joint distribution of $(X_k, X_{k+1})$ is the same as the
 distribution of $(X_{k+1}, X_{k})$ so that this is also often called
 the {\em reversibility condition}.  Another name often encountered is
 that $P$ is {\em $\lambda$-symmetric}.

 \subsection{Stability}
 \label{mcmc:sec:stability}
 MCMC applications typically are constructed so that a specific
 probability distribution $F$ is invariant.  However, in applications
 where MCMC is required it is typically difficult to simulate from the
 invariant distribution.  The most that can be hoped for is that the
 simulation will eventually produce a representative sample from $F$.
 This long-run behavior is in not guaranteed without additional
 assumptions. The following simple examples illustrate that the
 problems can arise due to the either the way the kernel is specified
 or the properties of the state space $\X$.

 \begin{example}
   \label{mcmc:ex:detmc}
   Suppose $F$ lives on $\{ 1, 2\}$ with $F(1) = 1 - F(2) = 1/4$ and
   \[
     P = \begin{pmatrix}
       0 & 1 \\
       1 & 0 \\
       \end{pmatrix}
   \]
   Since the Markov chain moves deterministically between the two
   states, it will over represent state 1 and underrepresent state 2
   no matter how many iterations there are.
 \end{example}

 \begin{example}
Suppose $F$ lives on $\{ 1, 2,3\}$ with $F(1) =  F(2) = F(3) = 1/3$ and
   \[
     P = \begin{pmatrix}
       1/2 & 1/2 & 0 \\
       1/2 & 1/2 & 0\\
       0 & 0 & 1
       \end{pmatrix}
     \]
     Thus starting at state $\{3\}$ the chain remains there forever
     while starting from $\{1, 2\}$ the chain will never visit
     $\{3\}$.  Thus the chain cannot represent $F$ in the long run.
 \end{example}
 

 \begin{example}
   \label{mcmc:ex:disc_Gibbs}
Suppose for $i=1,2$, $f_i$ is a pf on $\X_i \subseteq \real$ and $g_i$
is a pf on $\Y_i \subseteq \real$.
Set
\[
f(x,y) = \frac{1}{2} f_1(x) g_1(y) + \frac{1}{2} f_2(x) g_2(y).
\]
Then
\[
f_{X \mid Y}(x \mid y) = \frac{f_1(x) g_1(y) + f_2(x) g_2(y)}{g_1(y) + g_2(y)}
\]
and
\[
f_{Y\mid X}(y \mid x) = \frac{f_1(x) g_1(y) + f_2(x) g_2(y)}{f_1(y) + f_2(y)}
\]
and the Gibbs sampler MTD is
\[
k(x', y' \mid x, y) = f_{X \mid Y}(x' \mid y) f_{Y\mid X}(y' \mid x') .
\]
When $\X_i = \Y_i = \real$ this Gibbs sampler will produce a
representative sample eventually.  However, complications may arise if
the spaces are constrained.

Suppose $\X_1 = \X_2 = (0, 1)$ and that $f_1 = f_2$ is the
Uniform density.  Let $Y_1 = (0, 1)$ and $Y_2 = (2, 3)$ and $g_1$
and $g_2$ be Uniform densities.  Easy calculation yields that
\[
f_{X \mid Y} (x \mid y) = I(0 < x < 1) \left[ I(0 < y < 1) + I(2 < y <
3)\right]
\]
and
\[
f_{Y \mid X}(y \mid x) = \frac{1}{2} I(0 < x < 1) \left[ I(0 < y < 1) + I(2 < y <
3)\right]
\]
and that, no matter which square, $\X_1 \times \Y_1$ or $\X_2 \times
\Y_2$,  the current state is in there is a positive probability of the
next state being in either square.  This Gibbs sampler will eventually
produce a representative sample.

Now consider the setting with $\X_1 = \Y_1 = (0,1)$ and $\X_2 = \Y_2 =
(2,3)$ so that
\[
f_{X \mid Y} (x \mid y) = \frac{I(0 < x < 1) I(0 < y < 1) + I(2 < x < 3)I(2 < y <
3)}{I(0 < y < 1) + I(2 < y < 3)}
\]
and
\[
f_{Y \mid X} (y \mid x) = \frac{I(0 < x < 1) I(0 < y < 1) + I(2 < x < 3)I(2 < y <
3)}{I(0 < x < 1) + I(2 < x < 3)}.
\]
If $ y \in (0,1)$, then $f_{X \mid Y} (x \mid y) =  I(0 < x <1)$ and,
similarly, if $X \in (0,1)$, then $f_{Y \mid X} (y \mid x) = I(0 < y <
1)$.  Thus if the current state is in the square $\X_1 \times \Y_1$,
then the next step will be in $\X_1 \times \Y_1$.  That is, there is
no chance for the chain to visit $\X_2 \times \Y_2$.  This Gibbs
sampler will not produce a representative sample from the target
distribution.
\end{example}

The above examples demonstrate that one way problems arise is when the
Markov chain cannot access all of the space eventually and hence
properties which avoid this are required.

Let $\phi$ be a non-trivial positive measure on $\B$.  Then $A \in
\B$ is {\em $\phi$-communicating} if for all $B \subseteq A$ such that
$\phi(B) > 0$ and for all $x \in A$, there exists $n$ such that
$P^n(x,B) >0$. This is a weak property that does not alone ensure desirable long run
properties.  Consider the Gibbs sampler from
Example~\ref{mcmc:ex:disc_Gibbs} with $\X_1 = \Y_1 = (0,1)$ and
$\X_2 = \Y_2 = (2,3)$. If $\phi$ denotes Lebesgue measure, then this
Markov chain is $\phi$-communicating but does not have desirable long
run properties.

The Markov kernel, $P$,  is {\em $\phi$-irreducible} if for all $x \in \X$ and
for all $A \in \B$ such that $\phi(A) > 0$ there exists $n$ such that
$P^n (x, A) > 0$.  This is a key property in many MCMC settings, but
is not enough to ensure desirable long run properties; consider the
Markov chain in Example~\ref{mcmc:ex:detmc} which is irreducible.

There is some arbitrariness in the definition of
$\phi$-irreducibility, but if for some $\phi$, $P$ is
$\phi$-irreducible, then there exists a maximal irreducibility
measure $\psi$ (meaning that $\psi(A) = 0$ implies $\phi(A)=0$ for all
irreducibility measures $\phi$).

\begin{prop}
  \label{mcmc:prop:inv_irreducible}
  If $\lambda$ is an invariant measure for the Markov kernel $P$ and,
  for some $\phi$, $P$ is $\phi$-irreducible, then $P$ is
  $\lambda$-irreducible. 
\end{prop}

\begin{cor}
  If $P$ is $\lambda$-symmetric and $\phi$-irreducible, then $P$ is
  $\lambda$-irreducible.  
 \end{cor}

Often, inspection of the kernel $P$ and the underlying space $\X$
is enough to establish $\phi$-irreducibility.  For example, many kernels
obviously satisfy a positivity condition so that $P(x, A) > 0$ for
all $x \in \X$ and $A \in \B$ with $\phi(A) > 0$. The following is a
special case of a more general result \cite[][Theorem 3]{geye:2014},
but is often useful.

\begin{prop}
  \label{mcmc:prop:irred}
Suppose $\X$ is a connected, separable metric space.  If every
nonempty, open set $A$ satisfies $\phi(A)>0$ and every point has a
$\phi$-communicating neighborhood, the Markov kernel is
$\phi$-irreducible.  
\end{prop}

 
 
\section{Constructing MCMC Algorithms}
\label{mcmc:sec:construct}

\subsection{Metropolis-Hastings}
\label{mcmc:sec:mh}

{\color{blue} Give credit to R?}

The fundamental MCMC algorithm is Metropolis-Hastings \cite{hast:1970,
  metr:1953}.  It serves as a building block for many, if not most,
applications of MCMC.  The following will be generalized later, but,
for now, suppose $\X \subseteq \real^d$ and that $q(x,y)$ is a
proposal conditional density that is easy to sample.
%This notation can take an adjustment for those accustomed to the
%alternative $q(y \mid x)$, which is more common in statistics.
Define the {\em Hastings ratio}
\[
r(x,y) = \frac{f(y)q(y,x)}{f(x)q(x, y)} .
\]
%and note that the numerator defines a joint density for $(X,Y)$ while
%the denominator defines a joint density for $(Y,X)$.

\begin{algorithm}[H]
 \caption{Metropolis-Hastings} \label{mcmc:alg:mh}
 \begin{algorithmic}[1]
   \State {\it Input:} Current value $X_n = x$.

   \State Draw $Y \sim Q(x, \cdot)$ \State Draw
   $U \sim \text{Uniform}(0,1)$
   
   \State If $u \le r(x,y) \wedge 1$, accept $y$ and set $X_{n+1}= y$,
   else set $X_{n+1} = x$.
 \end{algorithmic}
\end{algorithm}

Algorithm~\ref{mcmc:alg:mh} is the formal definition, but is not at
all how the algorithm should be implemented in practice.  The
following implementation will help avoid overflow issues.

\begin{algorithm}[H]
 \caption{Metropolis-Hastings Implementation} \label{mcmc:alg:mh}
 \begin{algorithmic}[1]
   \State {\it Input:} Current value $X_n = x$.

   \State Draw $Y \sim Q(x, \cdot)$ \State Draw
   $U \sim \text{Uniform}(0,1)$

   \State If $\log r(x,y) \ge 0$ set $X_{n+1}= y$, else if $u \le
   r(x,y)$, set $X_{n+1}= y$, else set $X_{n+1} = x$.
 \end{algorithmic}
\end{algorithm}

Metropolis-Hastings defines a Markov kernel.  Specifically, if
$\alpha(x, y) = 1 \wedge r(x,y)$, then 
\begin{equation}
  \label{mcmc:eq:mh.kernel}
  P(x, \df y) = Q(x, \df y) + \delta_x(\df y) \int \left[ 1 -
    \alpha(x, u) \right] Q(x, \df u) .
\end{equation}

\begin{prop}
  \label{mcmc:prop:mh.symmetric}
  The Metropolis-Hastings kernel~\eqref{mcmc:eq:mh.kernel} is $F$-symmetric.
\end{prop}

\begin{proof}
  It suffices to consider $x \neq y$.  Then
  \begin{align*}
    F(\df x) P(x, \df y) & = f(x) q(x, y) \left[ 1 \wedge r(x, y)
                           \right] \mu(\df x) \mu(\df y) \\
    &  = \left[ f(x) q(x, y) \wedge f(y) q(y, x) \right] \mu(\df x)
      \mu(\df y) \\
    & = f(y) q(y, x) \left[ 1 \wedge r(y,x) \right] \mu(\df x) \mu(\df
      y) \\
    & = F(\df y) P(y, \df x) .
    \end{align*}
  \end{proof}

  The proof of the following result is easy and is left as
  an exercise.

\begin{prop}
 \label{mcmc:prop:mh.irreducible}
If $q(x, y) > 0$ for all $x,y \in \X$, then the Metropolis-Hastings
kernel $P$ is $F$-irreducible.
\end{prop}

\begin{example}
  Suppose continuous density $f$ has support $\X = \real$ and consider three settings.  Firstly, if the proposal
  distribution is the Student's $t$ distribution with $d \ge 1$
  degrees of freedom, then by
  Proposition~\ref{mcmc:prop:mh.irreducible} the resulting
  Metropolis-Hastings kernel is $F$-irreducible.    Secondly, if the proposal distribution is $\text{Uniform}(0,1)$, then the
  resulting Metropolis-Hastings kernel is obviously reducible since no
  value outside of $(0,1)$ will be proposed. Thirdly,  if $X_n = x$ and the proposal is $\text{Uniform}(x -1,
  x+1)$, then Proposition~\ref{mcmc:prop:irred} can be used to
  establish $F$-irreducibility.  The first two conditions of the
  proposition are easy consequences of the properties of the real
  line.  Let $0 < \delta < 1$ so that $N_{x, \delta} = (x - \delta, x + \delta)$ is a
  neighborhood of $\{ x\}$ and
  \[
P(x, N_{x, \delta}) = \int_{x-\delta}^{x+\delta} \frac{1}{2} \left( 1 \wedge
e^{-0.5(x^2 + y^2} \right) \df y > 0
\]
from which it is easy to see that $N_{x,\delta}$ is an $F$-communicating neighborhood.
\end{example}

\subsubsection{Classical Strategies for Choosing a Proposal}
\label{mcmc:sec:choosing.proposal'}

An {\em Independence Metropolis-Hastings} results if the proposal is
independent of the previous state.  That is, $Y \sim Q$.   For
example, suppose the proposal is a $d$-dimensional normal distribution
with fixed mean $m$ and covariance $C$, so that  $X_t' \sim N_{d}(m,
C)$.

Independence MH is one of the simplest and best understood MH Markov
chains.  However, it is unlikely to be effective in many settings, as will be
shown later.

Another well-studied version is the {\em Symmetric
  Metropolis-Hastings}, which results when the proposal distribution
which is symmetric about the current state $X_{t-1}$, so
$q(x, y)=q(y, x)$. For example, suppose the proposal is a
  $d$-dimensional normal distribution centered at the previous
  iteration with covariance matrix $h I_d$ so that
  $X_t' \mid X_{t-1} \sim N_{d}(X_{t-1}, h I_d)$.  The scaling
  parameter $h > 0$ is user-specified.


If the proposal of a symmetric Metropolis-Hastings also satisfies
$q(x, y) = q(\|x-y\|)$, then a {\em Random-Walk Metropolis-Hastings}
sampler results.  Note that the proposal,
$X_t' \mid X_{t-1} \sim N_{d}(X_{t-1}, h I_d)$, in the previous
example results in random walk Metropolis-Hastings.  Another example
would be a $\text{Uniform}(x - h, x + h)$, $h >0$, proposal.


Suppose the proposal distribution is a $d$-dimensional normal
distribution where the mean takes a gradient descent step and
covariance $h I_d$ so that
\[
  X_t' \mid X_{t-1} \sim N_d (X_{t-1} + (h/2) \nabla \log(f(X_{t-1})), h I_d) .
\]
The gradient does not require the normalization constant of $f$.
The motivation for this construction comes from discretized Langevin
dynamics \citep{robe:1996} and, consequently, is known as
Metropolis-Adjusted Langevin Algorithm (MALA).

These examples barely scratch the surface of Metropolis-Hastings
variations.  For example, much recent research has gone into
Hamiltonian Monte Carlo which uses a proposal based on discretized
Hamiltonian dynamics \citep{neal:2011}.  Extensions to Riemanian
manifold MALA and Hamiltonian variants also exist
\citep{giro:cald:2011}.  Some of these will be encountered later.

\subsection{Combining Markov Kernels}
\label{mcmc:sec:combining.kernels}

Markov kernels may be combined in order to create more effective
algorithms.  There are two basic ways of doing so that will be
exploited below.  For now, suppose $P_{1}, \ldots, P_{d}$ are Markov
kernels such that $FP_{i} = F$ for $i=1,\ldots,d$. The {\em
  composition} kernel is defined by
\[
P_C(x, \cdot) = (P_{1} \cdots P_{d})(x, \cdot) 
\]
and corresponds to cycling through the kernels in a specified
order. If each $r_i > 0$ such that $r_{1} + \cdots + r_{d} =1$, then
the {\em mixing} kernel is defined by
\[
  P_{m}(x \cdot) = r_{1} P_{1}(x, \cdot) + \cdots + r_{d} P_{d}(x, \cdot)
\]
and corresponds to updating via the kernel selected by the mixing 
probabilities $r_i$. It is an easy exercise to verify that $FP_{c} = F$ and $FP_{m} =
F$.

\subsection{Component-wise Updates}
\label{mcmc:sec:componentwise}

It is rare that Metropolis-Hastings can be used without modification in
practically relevant settings.  The distribution $F$ is often
too complicated or too high-dimensional for a block
Metropolis-Hastings update to be effective.  It is natural to seek to
work with smaller problems.

Since this is introductory, the focus will be on the setting where
$f(x,y)$ is a density function on $\X \times \Y$, that is, the
so-called {\em two-variable} setting.  The extension to the setting
with more than two variables is straightforward through the usual
properties of joint probability functions.

\subsubsection{Linchpin Variables}
\label{mcmc:sec:linchpin}

Let $f_{X|Y}$ be the pf of the conditional distribution
of $X$ given $Y$. Let $f_Y$ be the pf of the marginal
distribution of $Y$. If sampling from $f_{X\mid Y}$ is
straightforward, recall that $Y$ is a linchpin variable
 since
% 
\begin{equation}
  \label{eq:lv}
f(x,y) = f_{X \mid Y}(x|y)\, f_Y(y). 
\end{equation}
% 
Recall that exact samples can be obtained by first simulating $Y \sim f_{Y}$
followed by $X \sim f_{X \mid Y}$.  However, $F_{Y}$ may be complicated or difficult to sample directly so
it is natural to turn to MCMC.  Let $P_{Y}(y, \cdot)$ be a Markov
kernel such that $F_{Y}$ is invariant.  Then the linchpin variable
sampler is specified in Algorithm~\ref{alg:lvs}.

\begin{algorithm}
  \caption{Linchpin variable sampler} \label{alg:lvs}
  \begin{algorithmic}[1]
  \State {\it Input:} Current value $(X_j, Y_j)$
  \State Draw $Y_{j+1} \sim P_{Y}(Y_{j}, \cdot) $. 

  \State Draw $X_{j+1} \sim F_{X|Y}(\cdot \mid Y_{j+1})$.

  \State Set $j=j+1$
\end{algorithmic}
\end{algorithm}

The resulting Markov kernel is given by
\begin{equation}
  \label{mcmc:eq:lv_kernel}
P((x,y), A) = \int_{A} F_{X \mid Y} (dx' \mid y') P_{Y}(y, dy')
\end{equation}
That $F$ is invariant is left as an exercise.

Linchpin variable samplers have been employed in a variety of
scenarios. Their success is typically due to either (1) superior
mixing in the lower-dimensional space, (2) de-correlation of
components via the linchpin variables, or (3) lower post-processing
costs.


\subsubsection{Conditional Samplers}
\label{mcmc:sec:conditional}

\ref{mcmc:ex:2varGS}


\section*{Exercises}
\addcontentsline{toc}{section}{Exercises}
\begin{hw}
  \label{mcmc:hw:dbc}
Prove that if Equation~\ref{mcmc:eq:dbc} holds, then $\lambda$ is
invariant for $P$.
\end{hw}

%\begin{hw}
%What is the invariant distribution of the Markov chain defined in
%Example~\ref{mcmc:ex:detmc}?
%\end{hw}

\begin{hw}
  Consider the Gibbs samplers in Example~\ref{mcmc:ex:disc_Gibbs}.
  establish that the Gibbs sampler on $\X_1 = \Y_1 = (0,1)$ and
  $\X_2 = \Y_2 = (2,3)$ is not irreducible while the one on
  $\X_1 = \Y_1 = \X_2 = (0,1)$ and $ \Y_2 = (2,3)$ is irreducible.
\end{hw}

\begin{hw}
Establish Proposition~\ref{mcmc:prop:inv_irreducible}.
\end{hw}

\begin{hw}
  Establish Proposition~\ref{mcmc:prop:irred}.
\end{hw}

\begin{hw}
  Prove that Metropolis-Hastings  has Markov
  kernel given in~\eqref{mcmc:eq:mh.kernel}.  Hint: Note that if $A \in \B$
  and $j \ge 1$ then
  \[
    \Pr( X_{j+1} \in A \mid X_j) = \Pr(X_{j+1} \in A , U \le
    \alpha(x, y) \mid X_j) +  \Pr(X_{j+1} \in A , U >
    \alpha(x, y) \mid X_j) .
    \]
  \end{hw}


  \begin{hw}
    Establish that $F$ is invariant for  the linchpin variable kernel
    specified in \eqref{mcmc:eq:lv_kernel}.
  \end{hw}

  \begin{hw}
    Let $Y \in \real^n$, $\beta \in \real^p$, 
$u \in \real^k$, $X$ be a known $n \times p$ full column rank design matrix, and $Z$ a known $n \times k$ full column rank matrix. Also 
assume that $\max\{p, k\} <n$. Then a Bayesian
linear model is given by the following hierarchy
\begin{equation}\label{model1}
\begin{array}{rcl}
 Y|\beta, u, \lambda_E, \lambda_R & \sim & \textrm{N}_n\left(X\beta+Z u, \lambda_E^{-1}I_n\right) \\
 g(\beta)                            & \propto & 1 \\
 u|\lambda_E, \lambda_R           & \sim & \textrm{N}_k\left(0,\lambda_R^{-1}I_k\right) \\
 \lambda_E                        & \sim & \textrm{Gamma}\left(e_1, e_2\right) \\
 \lambda_R                        & \sim & \textrm{Gamma}\left(r_1,
   r_2\right) \,.
\end{array}
\end{equation}
Assume that $e_1, e_2, r_1, r_2>0$ are known hyper-parameters. This
hierarchy results in a proper posterior \cite{sun:etal:2001} . Let
$\xi = (\beta^T, u^T)^T$, $\lambda = (\lambda_E, \lambda_R)^T$ and let
$y$ denote all of the data.  Then the posterior density satisfies
% 
\begin{equation}\label{posterior-m22}
  f(\beta, u, \lambda_E, \lambda_R\mid y) = f(\xi, \lambda\mid y) =
  f_{\xi|\lambda}(\xi\mid \lambda, y) \, f_{\lambda}(\lambda\mid y)\,.
\end{equation}
    \end{hw}

\newpage

\section*{Appendix}
\addcontentsline{toc}{section}{Appendix}



\bibliography{../mcref}
\bibliographystyle{apalike}
\end{document}